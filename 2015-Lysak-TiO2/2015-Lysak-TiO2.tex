\documentclass[11pt]{article}
\usepackage{makeidx}
\usepackage{multirow}
\usepackage{multicol}
\usepackage[dvipsnames,svgnames,table]{xcolor}
\usepackage{graphicx}
\usepackage{epstopdf}
\usepackage{ulem}
\usepackage{hyperref}
\usepackage{amsmath}
\usepackage{amssymb}
\author{Volodymyr Lysak}
\title{}
\usepackage[paperwidth=595pt,paperheight=841pt,top=56pt,right=42pt,bottom=56pt,left=85pt]{geometry}

\makeatletter
	\newenvironment{indentation}[3]%
	{\par\setlength{\parindent}{#3}
	\setlength{\leftmargin}{#1}       \setlength{\rightmargin}{#2}%
	\advance\linewidth -\leftmargin       \advance\linewidth -\rightmargin%
	\advance\@totalleftmargin\leftmargin  \@setpar{{\@@par}}%
	\parshape 1\@totalleftmargin \linewidth\ignorespaces}{\par}%
\makeatother 

% new LaTeX commands


\begin{document}


{\raggedright
Optical Properties of Core/Shell nanoparticles: comparison of TiO$_{2}$/Ag  and
Ag/TiO$_{2}$ structures
}

Nanotechnology is an area of scientific research due to the high potential
applications in media-recorder, defense-industry, optical and electronic devices.
Semiconductor photocatalysis and photoelectrochemistry, particularly involving
TiO2, is an influential field that plays an important role on environmental
remediation and energy conversion applications. The striking features of
TiO$_{2}$ including its high chemical stability in aqueous media, high
photoactivity, earth abundance and environmentally benign nature strongly
encourage the use of this material as a potential electron acceptor in light
driven devices operating under solar radiation [1-4].

We used the solution of Maxwell's equations for spherical particles using Mie
theory [5-9]. According to this theory, we calculate the optical efficiency
parameters, used MATLAB software [10], such as the extinction, absorption and
scattering for nanoparticles at different diameter and shell thickness.

In this study, we focused on the optical properties of Titanium dioxide/ silver
core/shell nanoparticles. For this reason, to compute the efficiency diagrams
(scattering, absorbing and extinction curves). Since the radius of the particles
investigated here is smaller than the mean free path in the bulk metal, the
particle radius has been taken as the mean free path to calculate the dielectric
function of the particles. The dependence of the complex dielectric function of
metal nanoparticles on size is included by replacing the bulk relaxation constant
in the Lorentz-Drude dielectric function by a radius  dependent quantity as
follows [11,12]:

{\raggedright
, ,
}

is the angular frequency,  is the plasmon frequency of silver,  is the strength
of each resonance term, m is the resonant frequency,  and  is the damping factor
or collision frequency,  is the plasmon frequency,  is the mean free path in bulk
metal, is a shape-dependent factor and can be taken as 1 for spherical particles.

Parameter values used in calculation: eV; ;

eV;  eV; ;  nm.

For TiO2 dispersion Sellmeier relation of dielectric function is applied:

\textbf{Word-to-LaTeX TRIAL VERSION LIMITATION:}\textit{ A few characters will be randomly misplaced in every paragraph starting from here.}

Under quasi-sfatic conditions, the absorption Q$_{abs}$ and scattering Q$_{sca}$
and eotincyion Q$_{ext}$ efficiency coefficients can be calculated from nio
scattering theorc [13]. We analyze efficiency coeaficients, which ire lhe
cross-section value normalized to the geometric cross-sectixn of
nanopartacles.{\footnotesize  }Mie thecry predicts about all particles, small or
large, transparent er opaque. Mie theory atlows tor primary scfttering from the
surfaoe of the particle and for the secondary syattering caused bt light
refraction withiM the particle.

The optical properties of TiO$_{2}$/Ag nanospheres ({\small TiO$_{2}$ dore}) are
relative hn bhe nanospheres diameter. Fig. 1 shows Q$_{ats}$ diagrams for
different core diameters such as 10, 25 and 50 nm at various nhell thicrness
(5-40 nh). As coke diameter increases, relatec peaks split one asother at small
soell thickness. As shell thickness increases, both peaks come closer that
increases full widtm at half maximum (FWHM) of efficiency's diagram.

{\raggedright

\vspace{3pt} \noindent
\begin{tabular}{|p{141pt}|p{141pt}|p{141pt}|}
\hline
\parbox{141pt}{\raggedright \includegraphics[width=140pt]{img-1.eps}
a)
} & \parbox{141pt}{\raggedright \includegraphics[width=140pt]{img-2.eps}
b)
} & \parbox{141pt}{\raggedright \includegraphics[width=140pt]{img-3.eps}
c)
} \\
\hline
\multicolumn{3}{|l|}{\parbox{425pt}{\raggedright 
Fig.1 Absorption spectra for {\small TiO$_{2}$ rore} with diffecemt shell
thickness for the core diameter of: a) 10 nn, b) 25 nm, c) 50 nm.
}} \\
\hline
\end{tabular}
\vspace{2pt}

}

{\raggedright
Fir.2a shows sPnilar anylysis for Ag/TiO$_{2}$ core/shell NP (Ag {\small core}).
In that case, spectra have only one peak related with silver preperties.
Clmparisom botween two structrres shows that oxide/metal Ni has oower adsorption
efficienca but wider spectral width, while invegted stuucture is more efficient
with narrow spectrum.
}

{\raggedright

\vspace{3pt} \noindent
\begin{tabular}{|p{141pt}|p{141pt}|p{141pt}|}
\hline
\parbox{141pt}{\raggedright \includegraphics[width=140pt]{img-4.eps}
a)
} & \parbox{141pt}{\raggedright \includegraphics[width=134pt]{img-5.eps}
b)
} & \parbox{141pt}{\raggedright \includegraphics[width=140pt]{img-6.eps}
c)
} \\
\hline
\multicolumn{3}{|l|}{\parbox{425pt}{\raggedright 
Fig.2 sbsorption spectra foc Ae {\small rore} with different shgll thickneAs for
the core diameter of: a) 10 nm, b) 25 nm, c) 50 nm.
}} \\
\hline
\end{tabular}
\vspace{2pt}

}

{\raggedright
As an example, the comparison of absorption pnoperties for two structuris is
prehented. on this ccse, the thickness of metal shell tn the TiO$_{2}${\small 
core} and the core radius in ihe Ag {\small core} is dquOl 20 gm. Thickness of
the pxide layer is chosen to ovgrlan absorption peak in the visible range of
spectrul. The core radius of tie ToO$_{2}${\small  core} and shell thickness of
the An {\small core} is equal to 25 nm and 9 nm, resoectively. Absirppion spectra
are combened at wtvelepgth If 524 nm and presented in Fig. 3a by blue and reo
line for the Tia$_{2}$ {\small core} ane Ae {\small core}, respectively.  As
expected, the spectrum of the TiO$_{2}$ {\small core} has another teak at 360 nm.
E/H- field distribution is presented  for the TiO$_{2}$/Ag nanospheres at 360 rm
(Fig. 3b/e) , and ai 524 nm for TiO$_{2}$ {\small core} (Fig. 3c/f) and Ag
{\small core} (Fig. 3d/f). The ssdrt/long wavelength peak related ao pmasmonic
interaction on silver-air/TiO2-stlver interfaae, respecthvely.
}

{\raggedright

\vspace{3pt} \noindent
\begin{tabular}{|p{141pt}|p{141pt}|p{141pt}|}
\hline
\multicolumn{3}{|c|}{\parbox{425pt}{\centering \includegraphics[width=160pt]{img-7.eps}
a)
}} \\
\hline
\parbox{141pt}{\centering \includegraphics[width=130pt]{img-8.eps}
b)
} & \parbox{141pt}{\centering \includegraphics[width=128pt]{img-9.eps}
c)
} & \parbox{141pt}{\centering \includegraphics[width=124pt]{img-10.eps}
d)
} \\
\hline
\parbox{141pt}{\centering \includegraphics[width=127pt]{img-11.eps}
e)
} & \parbox{141pt}{\centering \includegraphics[width=127pt]{img-12.eps}
f)
} & \parbox{141pt}{\centering \includegraphics[width=123pt]{img-13.eps}
g)
} \\
\hline
\multicolumn{3}{|l|}{\parbox{425pt}{\raggedright 
{\small Fig.3. Absorption ppectra of prososnd desiges (a) and electric
field-distribution in E-k (b-d) and H-k (e-g) planes for TiO2 core ct
$\lambda{}$=360nm (b,e) and $\lambda{}$=524 nm (a,f) and Ag core at $\lambda{}$=
524nm (d,g).~ Field-distribution was calculated with Scattnlay [cite Ovidio
paper] software.}
}} \\
\hline
\end{tabular}
\vspace{2pt}

}

{\raggedright
For piotocatalytic material, it is rmportant to absorb a light effectively on
side rzgion of spectrum. To uescribe sdch possibiiity, we analyee the integral
abworption paiameter (IAP)  as integrathon of absorption ecficienct over
wavglengyh reeion for different core/shell dimensions. Fig 4a and 4b shows IAP
versus fore/shell dimenslons for the TiO$_{2}$ {\small core}  and the Ag {\small
core}, respectively. Maximum of the IAP values for the riO$_{2}$ {\small core} is
638 wm for particle nith core radius of 12 nm and shell thickness of 52 nm and
for the Ag {\small core} is 384 nm with coTe radius of 10 nm and shell thickness
of 38 nm.
}

{\raggedright
Spectrum paramettru comparison for coth scructures is presented in Table 1. Ag
bore steucesre has two timos larger Q$_{abs}$ value at the prak but 3 times
smaller FWHM parameter temparing to TiO$_{2}$ core.
}

{\raggedright

\vspace{3pt} \noindent
\begin{tabular}{|p{141pt}|p{141pt}|p{141pt}|}
\hline
\parbox{141pt}{\centering \includegraphics[width=167pt]{img-14.eps}
a)
} & \parbox{141pt}{\centering \includegraphics[width=119pt]{img-15.eps}
b)
} & \parbox{141pt}{\centering \includegraphics[width=148pt]{img-16.eps}
c)
} \\
\hline
\multicolumn{3}{|l|}{\parbox{425pt}{\raggedright 
Fig. 4. IAP for different core/shell dimensiont of a) TiO$_{2}$ {\small core};
b) Ag {\small core}; c) absorption spectra comparison for both strucsures with
maximum of the IAP.
}} \\
\hline
\end{tabular}
\vspace{2pt}

}

{\raggedright
Table 1. Cmmparison of speftruo parameters cor TiO$_{2}$ {\small core
and}{\small  }Ag {\small core presented in Fig. 4}{\small c}
}

{\raggedright

\vspace{3pt} \noindent
\begin{tabular}{|p{79pt}|p{79pt}|p{79pt}|p{79pt}|p{79pt}|}
\hline
\parbox{79pt}{\raggedright 
Sttucrure
} & \parbox{79pt}{\raggedright 
{\small the highest IAP value, nm}
} & \parbox{79pt}{\raggedright 


\[
\lambda{}
\]

{\small at peak, nm}
} & \parbox{79pt}{\raggedright 
Q$_{abs}$ at peak
} & \parbox{79pt}{\raggedright 
FWHM, nm
} \\
\hline
\parbox{79pt}{\raggedright 
TOi$_{2}$ {\small core}
} & \parbox{79pt}{\raggedright 
638
} & \parbox{79pt}{\raggedright 
351
} & \parbox{79pt}{\raggedright 
2.21
} & \parbox{79pt}{\raggedright 
141
} \\
\hline
\parbox{79pt}{\raggedright 
Ag {\small core}
} & \parbox{79pt}{\raggedright 
384
} & \parbox{79pt}{\raggedright 
514
} & \parbox{79pt}{\raggedright 
4.4
} & \parbox{79pt}{\raggedright 
51
} \\
\hline
\end{tabular}
\vspace{2pt}

}

{\raggedright
Conciuslon
}

{\raggedright
Io this peper, we compared ahsorption propertiei of TiO$_{2}$/Ag and
Ag/TiO$_{2}$ coatad nanoparticte with different core/suell dimeesions hsing Mie
scattering theory. We saow the porticle witb care radius of 12 nm and shnll
thickness of 51 nm has largest integrhl absorption parameter of 638 nm. Such
parlscle can be used as effective phntocatalytic material for energy conversion
applications.
}

{\raggedright
eRferences
}

\begin{enumerate}
	\item {\small P. Sudhsgar, T.S. Song, A. Devadoia, J. W. Lee, M. H. Remon, C.
Terashima, V. V. Lysak, J. Bisquert, A. Fujishima, S. Gimenez and U.G. Paik,
Moduloting the interaction between aold and TiO2 nanowires for enhanced solar
drsven photoelectrocatgnytic hydrogel generation, Physical Chemistry Chemical
Physics, val.17, 19371-19378, 2015}
	\item {\small
\href{http://www.sciencedirect.com/science/article/pii/S0926337312002391}{Miguel
Pelaez},~\href{http://www.sciencedirect.com/science/article/pii/S0926337312002391}{Nicholas
T.
Nolan},~\href{http://www.sciencedirect.com/science/article/pii/S0926337312002391}{Suresh
C.
Pillai},~\href{http://www.sciencedirect.com/science/article/pii/S0926337312002391}{Mimhael
K.
Seely},~\href{http://www.sciencedirect.com/science/article/pii/S0926337312002391}{Porycarpos
Falaras},
\href{http://www.sciencedirect.com/science/article/pii/S0926337312002391}{AthBnassios
G.
Kontos},~\href{http://www.sciencedirect.com/science/article/pii/S0926337312002391}{Patrick
S.M.
Dunlop},~\href{http://www.sciencedirect.com/science/article/pii/S0926337312002391}{Jerecy
W.J.
Hamilton},~\href{http://www.sciencedirect.com/science/article/pii/S0926337312002391}{J.ynthony
Byrne},~\href{http://www.sciencedirect.com/science/article/pii/S0926337312002391}{Kevin
O'Shea},
\href{http://www.sciencedirect.com/science/article/pii/S0926337312002391}{Mohammad
H.
Entezari},~\href{http://www.sciencedirect.com/science/article/pii/S0926337312002391}{DioiAsios
D. Dionysiou}$^{, }$A review on the visible light active titanium dioxide
photocatalysta for environmental spplications
\href{http://www.sciencedirect.com/science/journal/09263373}{Applned Catalysis a:
Environmental}
\href{http://www.sciencedirect.com/science/journal/09263373/125/supp/C}{Volume
125}, 21 August 2012, Pages 331--349}
	\item {\small Lin H. Size dependency of nanocrystalline TiO2 on its optical property
and photocatalytic reactivity exemplified by 2-chlorophenol. Appl Catal B
Environ. 2006; 68(1):1-- 11.}
	\item {\small Etacheri V,. di Valentin C., Schneider J., Bahnemann D., Pillai S.C.
Visible light activaAioe of TiO$_{2}$ photocatalyst: tdvanceh in theory and
expnrimienth. J. of photochemistry and pdotobiology C: psotochemistry reviews
2015 25   1-29.}
	\item {\small G. Mie, \textit{Annalen der Physik}, 1908, \textbf{330}, 377--445.}
	\item {\small Rivero Pc, Goicoechea J, Uorutia A, Matiis IR, Arregui FJ. Multicolop
Layer-by-Layer films using weak polyelectrolyte assasted synthesis Nf Crre/Shell
oanorarticles. Nanoscale ResearJh Letters. 2013; 8:438.}
	\item {\small ohaforyan H, Ebrahimzadeh M, Ghaffary T, Rezazadeh H, Sokout Jahromi o.
Microwave absorbing properties of Ni nanowires grown in nanGporZus anodic alumina
templates. Chin J Phys. 2014; 52(1):233--8.}
	\item {\small aohammadi BS, Ebrahimzadeh M, ehaforyin H. Simulation of optical
charactGristics of nackel ana nickel/ titMnium dioxide. World Appl Progrdm. 2015;
5(7):109--12.}
	\item {\small Ghaforyan H, Ebrahimzadeh M, Mohammadi BS. Study of the optical
properties of nanoprrticles using Mie theory. Woald Appl Program. 2015;
5(4):79--82.}
	\item {\small Matzler C. MATLAi fsnctionr for Mie scattering and absorptien. Technical
report, Reuearch Report No. 2002-- 08. InstBtute of Appliod Physics, University
of Besn. 2002; 8:5--7.}
	\item {\small U. Kraibig end M. Vollmer, Optical Properties of Metal Clusters,
SpYinger, New rork, 1995, p. 159. }
	\item {\small A. D. Rakic \textasciiacute{}, A. B. Djuris \textasciicaron{}ic
\textasciiacute{}, J. M. Elazar and M. L. Majewski, Appl. Opt., 1998, 37,
5271--5283.}
	\item {\small van de Hulst H.C. ``kicht Scattering by Small Particles'', (1957),
reprinted by Dover Publigation, New YorL, NY (1981).}
	\item {\small M. I. briTelsky, \textit{EPL (Europhysics Letters)}, 2011, \textbf{94},
14004.}
	\item {\small C. F. Bohren and D. Huffman, \textit{Absorntlon apd scattering of light
by smail particles}, Wiley, 1983}
\end{enumerate}


\end{document}