%\documentclass[aps,prl,twocolumn,showpacs,superscriptaddress,groupedaddress]{revtex4-1}  % for review and submission
%\documentclass[aps,preprint,showpacs,superscriptaddress,groupedaddress]{revtex4-1}  % for double-spaced preprint

%Phys Rev B
%Letter or rapid communication - 3500 words
%\documentclass[prb,groupedaddress,reprint]{revtex4-1}
%APL
\documentclass[aps,prl,twocolumn,showpacs,superscriptaddress,groupedaddress]{revtex4-1}
%\documentclass[aps,prl,reprint]{revtex4-1}
%\documentclass[aip,jap,reprint]{revtex4-1}
\usepackage[english]{babel}
\usepackage{graphicx}% Include figure files
\usepackage[pdftex,unicode,colorlinks, citecolor=blue,%
filecolor=black, linkcolor=blue, urlcolor=black]{hyperref}
\usepackage[figure,table]{hypcap} %links should lead to the begining
                                %of figure or table...

\begin{document} %Max size in pdf for APL - 4 print pages

\title{Super absorption and efficient absorption of light by spherical
  nanoparticles}


\author{Konstantin Ladutenko} \email[e-mail: ]{fisik2000@mail.ru}
\affiliation{ITMO University, 49 Kronverskii Ave., St.~Petersburg
  197101, Russian Federation\\}

\affiliation{Ioffe Physical-Technical Institute of the Russian Academy
  of Sciences, 26 Polytekhnicheskaya Str., St.~Petersburg 194021,
  Russian Federation}

\author{Ovidio Pe\~{n}a-Rodr\'{i}guez} \affiliation{Instituto de
  Fusi\'{o}n Nuclear, Universidad Polit\'{e}cnica de Madrid,\\
  Jos\'{e} Guti\'{e}rrez Abascal 2, E-28006 Madrid, Spain}


% \author{Pavel Belov} \affiliation{ITMO University, 49 Kronverskii
% Ave., St.~Petersburg 197101, Russian Federation\\}
\author{Ali Mirzaei} \author{Andrey Miroshnichenko} \author{Ilya
  Shadrivov} \affiliation{Nonlinear Physics Centre, Research School of
  Physics and Engineering, The Australian National University, 59
  Mills Rd, Acton, ACT, 2601, Australia}

\date{\today}
% APL 250 words! Phys Rev B less then 500 words, about
% 5% ot total paper length
% emacs M-x count-words

\begin{abstract}
  % There is a theoretical limit for absorption by a sub-wavelength bulk
  % spherical particle.  To overcome this limit we applied a widely used
  % ``super'' design pattern which superpose several electric and
  % magnetic multipole resonances of a multilayered particle.  We used a
  % straightforward approach to evaluate a number of designs from
  % realistic materials.  However, we found that due to dimension effect
  % it can be preferable to use a properly designed smaller particle
  % with only a dipole response in order to reach the best absorption
  % efficiency.
  Spherical subwavelength nanoparticles have a fundamental limit as to
  how much light they can absorb. This limit is based on the
  assumption that only one mode is excited in the nanoparticle. Using
  stochastic optimization algorithm, we design multi-layer nanoparticles, in which we
  can make several resonant modes overlap at the same frequency, thus
  significantly beating the theoretical limit of absorption, and we
  call this super-absorption. We further introduce the efficiency of
  the absorption for a nanoparticle, which is absorption normalized by
  the physical size of the particle, and show that efficient absorbers
  are not always operating in super-absorbing regime.
\end{abstract}


\pacs% insert suggested PACS numbers in braces on next line
{41.20.Jb 42.25.Bs 02.60.Pn 02.70.-c}
% 41.20.Jb Electromagnetic wave propagation 42.25.Bs Wave propagation,
% transmission and absorption
%% 42.25.Fx Diffraction and scattering
% 02.60.Pn Numerical optimization 02.70.-c Computational techniques;
% simulations

\maketitle %\maketitle must follow title, authors, abstract and \pacs

Mie theory~\cite{Mie-1908}, which is over 100 years old now, describes
interaction of electromagnetic waves with spherical particles. Mie
solution is still of great interest these
days~\cite{Suzuki-2008,MacKowski-2012,Lerme-2000,Xu-2005,Li-2006,Gogoi-2010,Santiago-2011},
since it is one of the primary tools for analyzing wave scattering by
spherical objects. Further development of the Mie
theory~\cite{Yang-2003, Pena-scattnlay-2009} made it possible to apply
study the multilayered spherical
particles~\cite{Sheehan-2013,Selmke-2012}.  Such particles have
various applications in cancer treatment~\cite{Zhang-2010,
  Hirsch-2003} and medical diagnostics~\cite{Allain-2002},
cloaking~\cite{Qui-2009, Semouchkina-2013, Ladutenko-2014} and
plasmonic~\cite{Martin-2013, Alu-2005} devices, study on thermal
properties of insulating material~\cite{Xie-2013}, as well as for
improving solar cells performance~\cite{Kameya-2011,Mann-2011}, and so
on.

The scattering properties of multilayer cylinders and spheres was
studied in great detail by Fan~et~al.~\cite{Fan-2010,Fan-2011}.  In these
works authors introduced the concept of a super scattering, when the
scattering cross-section of a multilayer particle exceeds that of a
homogeneous particle of the same size in the so-called single-channel
limit. The super scattering appears when we design a multilayer
structure so that several modes become nearly degenerate, i.e. their
resonance frequencies coincide or get close to each other. In a
homogeneous particle, the resonances appear at different frequencies,
and there is no design freedom for a fixed geometry of the structure
to make these resonances overlap, and this limits the achievable
scattering cross-section.

Similar fundamental limitations exist for the absorbing properties of
sub-wavelength nanoparticles.  Tribelsky~\cite{Tribelsky-2011} has derived a
theoretical limit of a maximum absorption cross-section (ACS) value
for a single channel, i.e., when only one mode of the sphere is
excited.  As a result the absorption coefficients $\tilde{a}_n= {\rm
  Re}\{a_n\} - |a_n|^2 $ and $\tilde{b}_n= {\rm Re}\{b_n\} - |b_n|^2 $
become limited by $1/4$, here $a_n$ and $b_n$ are scattering
coefficient as defined in Mie theory~\cite{Bohren-1983}.

To overcome these limitations, we employ similar approach for
enhancing absorption cross-section~\cite{Fan-2011}. In particular, we
propose to use the multi-layer structures, and by means of genetic
algorithm we optimize the ACS of these particles. We analyze the
absorption cross-section of these particles, and present the
super-absorption regime. We further introduce the absorption
efficiency, which is the ACS normalized to the geometric cross-section
of the particles. Here we show that there is a strong counter-play
between the increased absorption for larger particles vs size for
smaller particles, and quite remarkably we find that the most {\em
  efficient} absorption can be achieved in a single channel limit for
small particles.

Another approach was given in a recent work of Grigoriev et
al.~\cite{Grigoriev-2015} with expressions for ideal absorber;
however, they considered only a dipole approximation, the final value
for our range of particle sizes is very close to the dipole limit
predicted with Tribelsky~\cite{Tribelsky-2011}.  This way
super-absorption designs are out of scope from the Grigoriev ideal
case; moreover, Grigoriev also provide an equation to desing a
core-shell structure from predefined materials. However, in case of
material parameters for $Si$ and $Ag$ and size parameter
from our best design, Grigoriev`s equation gives a complex value for
relative share of two layers, which is not suitable for most of the
simulation software or experiment.


\begin{figure}
  \center{
    % Use pdfcrop to remove white margins
    \includegraphics[width=0.4\textwidth]{fig/Fig1}%
    \caption{ Shematic view of the simulated $Si/Ag/Si$ particle. 
      \label{fig:geom}
    }%
  }
\end{figure}


We start our analyses by considering the triple layered $Si/Ag/Si$
spherical particle illuminated with a plane wave~(Fig.~\ref{fig:geom}). In what
follows we describe the materials using experimentally measured
parameters from the Ref.~\cite{palik-1997}.  To optimize the thickness of each
layer we implemented~\cite{JADE-web} an adaptive differential
evolution algorithm~\cite{Storn-DE-first-1997}, which is called
JADE~\cite{Jingqiao-JADE-2009}.  The technical details of the
optimization algorithm are published previously in
Ref.~\cite{Ladutenko-2014}. We perform Mie calculations using the
Scattnlay~\cite{Pena-scattnlay-2009,Scattnlay-web} software, whose results were
verified with a number of other implementations of the Mie solutions
and with a commercially available software such as CST Microwave
studio~\cite{CST-web} and Comsol Multiphysics~\cite{Comsol-web}.

It is obvious that in general, a bigger particle will have a bigger
absorption cross-section, so sphere with the diameter of 1 cm will
absorb more light than any sphere at the nanoscale. Therefore, we use
absorption efficiency $Q_{\rm abs}=C_{\rm abs}/2\pi R^2$,
where $R$ is the outer radius of the particle and $C_{\rm
  abs}$ is the absorption cross-section. We maximize absorption
efficiency at a fixed wavelength of incident light (we have chosen
$\lambda=500$~nm).

\begin{figure}
  \center{
    % Use pdfcrop to remove white margins
    \includegraphics[width=0.4\textwidth]{fig/2015-04-01-Qabs-SiAgSi-overview}%
    \caption{ Results of the optimization of the absorption efficiency
      for the fixed wavelength of 500~nm. (a) Absorption efficiency
      with the best value achieved at the particle radius of 36~nm and
      Ag/Si design (zero sized core). Dashed lines show theoretical
      limits for the first channel and second channel
      absorption. Second and third peaks in the absorption efficiency
      curve exceed the theoretical limit for the second mode
      absorption at $R=63$~nm and $R=81$~nm. (b) Mie absorption
      coefficients for individual excited modes of the optimized
      structures. (c) Optimized layer thicknesses. For the total
      particle radius below 46~nm the optimizer converges to
      the two-layer structure, when core size vanishes, and the
      optimum design is a bi-layer $Ag/Si$ particle. 
      \label{fig:overview}
    }%
  }
\end{figure}
%
In order to study the dependence of the absorption efficiency on the
outer particle size, we run optimization algorithm for different
(fixed) particle outer size, and our optimization parameters are the
radii of internal cores, whereas the target function is the absorption
efficiency.  We show the results of our genetic optimization algorithm
in Fig.~\ref{fig:overview}~(a).  Dashed lines show theoretical absorption
limit of a dipole ($n=1$) and a quadrupole ($n=2$)
resonances~\cite{Tribelsky-2011}, which are given as $$Q^{(n)}_{\rm
  abs\ max}=\frac{2n+1}{2q^2},$$ where the size parameter $q=2\pi
R/\lambda$, and $n$ is an angular momentum of the mode. Following
Ref.~\cite{Fan-2011}, where authors introduce super scattering for
spherical particles, here we we introduce super-absorption, when the
ACS is larger than the theoretical limit for absorption by the mode
with highest excited angular momentum $n$. In our parameter space we
have just modes up to the quadrupole excited ($n=2$), and in order to
get a super-absorption our efficiency should be higher than that of a
quadrupole. We clearly see this super-absorption at $R>60$~nm, in
Fig.~\ref{fig:overview}~(a).

In Fig.~\ref{fig:overview}~(b) we present the values of Mie absorption
coefficients for individual excited modes in the structure, while
horizontal dashed line shows the theoretical limit (1/4) for each of
them. $\tilde{a}_{1,2}$ are electric dipole and electric quadrupole,
while $\tilde{b}_{1,2}$ are magnetic dipole and magnetic
quadrupole. For small particles, as expected, the absorption is
dominated by an electric dipole $\tilde{a}_1$.  At $R >
56.6$~nm the optimization procedure finds that the designs with both
electric and magnetic dipoles have larger ACS, than the structure with
only the electric dipole excited. This is why the curves in
Figs.~\ref{fig:overview}~(b,c) experience the discontinuity. We also
note that there is a very narrow range of particle sizes, between
80.7~nm and 82.1~nm, where our analyses finds that the design
supporting electric dipole $\tilde{a}_1$ and magnetic quadrupole
$\tilde{b}_2$, has larger ACS, and this explains two more
discontinuities of the curves at the respective size values.

Fig.~\ref{fig:overview}~(c) shows optimized sizes of the layers inside
the multi-layer structure. It reveals quite a curious result, that the
dipole branch (i.e. for particle radii below 56.6~nm) has two
parts. For $R<46$~nm the best absorber has just two
layers, as the radius of the core of the three-layer structure
vanishes, and the particle reduces to $Ag/Si$ core-shell structure.
At $R=46$ dipole channel becomes practically
undistinguishable from the theoretical limit (it becomes
$\tilde{a}_1>0.249$).  It appears that the optimizer introduced the
inner $Si$ layer in order to keep $\tilde{a}_1$ near the theoretical
limit as the $R$ increases.  As a side effect, the
quadrupole contribution $\tilde{a}_2$ appears, however, it does not
help to reach super-absorption limit $n=2$.

Remarkably, the absolute maximum absorption efficiency is not reached
within the super-absorption regime. Figure~\ref{fig:overview} shows
that the maximum efficiency is reached for small particle size, and
the ACS is still well below the single channel limit. It appears that
the $Ag/Si$ core-shell nanoparticle, with the total radii of
approximately 36~nm is the most efficient absorbing in the considered
structure, which ACS reaching values over 5 times the physical
cross-section area of the particle.  From practical point of view it
is quite important that the maximum can be reached in a bi-layer
structure, instead of a triple-layer, and it should be easier and
cheaper to produce.

\begin{figure}
  \center{
    % Use pdfcrop to remove white margins
    \includegraphics[width=0.4\textwidth]{fig/2015-04-01-SiAgSi-ab-spectra3.pdf}%
    \caption{Spectra of Mie absorption coefficients of (a) efficient and (b-c)super absorption design.      
      \label{fig:spectra}
    }%
  }
\end{figure}
To study spectral properties of the structures with large ACS which we
obtained by the optimization, in Fig.~\ref{fig:spectra} we plot three
different cases for designs that correspond to local maxima of $Q_{\rm
  abs}$ shown in Fig.~\ref{fig:overview}~(a).  As expected, the design
corresponding to the maximum absorption efficiency at $R=36$~nm has a
single electric dipole resonance centered at the target wavelength
$\lambda=500$~nm. Spectra of designs with maxima at $R=63$~nm and
$R=81$~nm have a signature of the super absorption, i.e. there is an
overlap of several resonances.  We note that these structures have
additional absorption resonances, but they are located far from the
wavelength of interest.

A noticeable feature of Fig.~\ref{fig:spectra}~(c) is an almost flat top
of the electric dipole resonance.  We do not have any good explanation
%maybe there are two different
%dipole "types", and we see their interference?
for this phenomena, however, we can suggest that it is due to coupling
of the electric dipole $\tilde{a}_1$ with the magnetic quadrupole
$\tilde{b}_2$.  Whereas resonant responses of other desings originate
from coupling of incident plane wave with the corresponding multipole,
it is not the case for the design with $R=81$~nm. Here, the particle is
mostly composed from the $Si$ outer shell, that give enough volume for
$\tilde{b}_2$.  Dipole response comes from the small inner part of the
sphere and can not be directly exited with the incident wave.  It
takes the power from the surrounding $\tilde{b}_2$ mode and very soon
it reaches the fundamental limit for absorption.  This way we can
observe a flat top for the $\tilde{a}_1$ response accompanied with a
sigificat value of $\tilde{b}_2$.  Our suggestion is indirectly proved
by the reduced width of the $\tilde{b}_2$ resonance compared to other
quadrupole responses.

\begin{figure}
  \center{
    % Use pdfcrop to remove white margins
    \includegraphics[width=0.45\textwidth]{fig/SiAgSi-flow-R62-YZ-Eabs.pdf}%
    \caption{ Amplitude of electric field for $R=36$~nm (a,c) and
      ``super'' (b,d) designs in E-k  (a-b) and H-k (c-d) planes. 
      \label{fig:field}
    }%
  }
\end{figure}
Finally, we present distribution of the amplitude of the electric
field in Fig.~\ref{fig:field} for two designs: with the best
efficiency at $R=36$~nm and in a super-absorbing design with
$R=63$~nm.  We also plot streamlines of the Poynting vector which characterize
energy flow. For the effective design of the absorber, the power from
a large cross-sectional area flows into the particle.  In case of
super-absorption regime, we observe power flow vortices, which make
absorbtion more efficient as the electromagnetic energy propagates
several times through the absorbing materials.

In conclusion, we find the effect of the super-absorption, when the
absorption cross-section of the nanoparticle can exceed the
theoretical limit for the absorption by the highest excited mode. This
occur when several resonance modes of the structure overlap at the
same frequency. We introduce efficiency of the absorption as an
absorption cross-section divided by a geometric cross-section of the
particle, and quite unexpectedly we find that the most efficient
absorbers are smaller nanoparticles working in a single mode
regime. We present their spectral characteristics and field structure.

%TODO 
WE NEED TO WRITE ABOUT THIS AND SOME OTHER WORKS IN THE
INTRODUCTION. THE TOPIC IS DONE TO DEATH. 
It is interesting, that similar conclusion was made by Miller et
al.~\cite{Miller-2014} for extinction of arbitrary particles: small
size with only dipole responce is preferable for geometric volume
normalized efficiency.


\begin{acknowledgments}

  TODO acknowledgments
  % The authors would like to thank the Ministry of Education and
  % Science of the Russian Federation (Goszadanie 2014/190, Zadanie
  % No. 3.561.2014/K, project 14.584.21.0009 10), Russian Foundation
  % for Basic Research (Grant 15-02-01344), and Government of the
  % Russian Federation (Grant 074-U01) for the financial support. The
  % simulations of cloak designs has been funded by the Russian
  % Science Foundation Grant No. 14-12-01227.% We
  %   % would also like to appreciate Alexander Krasnok for the help
  %   % with
  %   % CST simulations and Ilya V. Shadrivov for valuable comments on
  %   % the
  %   % manuscript.
\end{acknowledgments}

\bibliography{2015-Ladutenko-Qabs}
\end{document}

